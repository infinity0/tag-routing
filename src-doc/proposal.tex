\documentclass[12pt]{article}
\usepackage{a4wide}

\parindent 0pt
\parskip 6pt

\begin{document}

\thispagestyle{empty}

\rightline{\large\emph{Ximin Luo}}
\medskip
\rightline{\large\emph{Queens' College}}
\medskip
\rightline{\large\emph{xl269}}

\vfil

\centerline{\large Computer Science Tripos Part II Project Proposal}
\vspace{0.4in}
\centerline{\Large\bf Tag-driven routing in a distributed storage network}
\vspace{0.3in}
\centerline{\large 2009-10-30}

\vspace{0.5in}

{\bf Project Originator:} self

\vspace{0.1in}

{\bf Resources Required:} none

\vspace{0.5in}

{\bf Project Supervisor:} \emph{George Danezis}

\vspace{0.2in}

{\bf Signature:}

\vspace{0.5in}

{\bf Director of Studies:} \emph{Robin Walker}

\vspace{0.2in}

{\bf Signature:}

\vspace{0.5in}

{\bf Overseers:} \emph{Alan Blackwell}\ and \emph{Ian Wassell}

\vspace{0.2in}

{\bf Signatures:}

\vfil
\eject

\section*{Introduction}

The search capabilities of most current peer-to-peer networks are weak compared
to centralised indexes on the world wide web. Popular query-routing algorithms,
and their main shortcomings, include:

\begin{description}
\item [Broadcasting:] (eg. old Gnutella) These are generally susceptible to
  flooding attacks and do not scale well.
\item [Multicasting, random walks:] (eg. OneSwarm) The reach of the request is
  limited; there is no systematic way to obtain results beyond this limit.
\item [Routing hashkey-addresses:] (eg. DHTs, Freenet, GNUnet) This requires
  prior knowledge of the target object. Some systems emulate a namespace by
  converting keywords to sets of hashkey addresses (eg. GNUnet {\tt KBlocks}),
  but this is hard to secure whilst remaining globally open.
\end{description}

The aim of this project is to build a basic high-level search system for a
distributed storage network, that addresses the above shortcomings. It will use
semantic routing to find objects from anywhere on the network, hopefully with
good probability even if the object is rare.

In order to perform semantic routing, the system will make use of the following
data structures. (Conceptually, these are only abstract objects; the mechanism
in which they are stored and retrieved is, for now, overlooked.)

\begin{description}
\item [index:] an inverted index that pins tags against objects. The objects
  could be content (ie. search results), but they could also be other indexes.
\item [tag-graph:] a graph that defines relationships between tags. The design
  must allow an algorithm to determine whether, and to what extent, a tag is
  a more "general" or more "specific" form of another tag (in some sense).
\end{description}

Any node will be able to publish such objects, and to give approval to the ones
it encounters. The routing algorithm would then involve retrieving trusted
indexes and tag-graphs, and efficiently navigating the network of links between
them. The structure of the network is likely to greatly affect its performance;
this will be explored in the evaluation.


\section*{Design issues}

I have chosen to focus on tag-based searches, since the vast majority of global
queries are of this level of complexity. Generally, advanced queries only need
to operate on a local, rather than global scale.

Indexes could (and ideally would) be automatically generated; however, methods
of assigning tags to data is a whole topic in itself and this project will not
cover it. Instead, I will assume that tags have already been assigned, and work
on designing a system to deliver this information globally.

Allowing anyone to maintain indexes has the advantage of not limiting metadata
supply to a small minority (eg. the authors), who could abuse their position,
or just be inefficient. Experience indicates that this tends to give better
results, even if the "metadata" is as simple as a link.

Hierarchical address spaces are good for routing. With semantic tags, however,
research shows that strict hierarchies are hard to maintain, and eventually
become inconsistent and inadequate for usage needs. So we take the basic idea
behind hierarchical routing, and modify it slightly: instead of having just one
parent, any tag can now be semantically related to any number of more "general"
tags. This relationship creates a tag-network that contains subgraphs with
hierarchy-like structures, to guide routing.

A pattern similar to this is mirrored in society: more "general" ideas are
known by a greater subset of the population to a lesser degree of detail, with
the gaps often being filled instead by pointers to people who do know the
details. As humans, we often exploit this property in our everyday attempts to
find information. We can use the index and tag-graph data structures to
construct a network with this property, and then derive a routing algorithm
that exploits it to perform efficiently.

Extracting search results then becomes a question of which indexes to trust,
and how to efficiently traverse the links between indexes to reach the data you
want. We can use social network information in order to facilitate this, such
as having nodes give (revokable) approval ratings for indexes and tag-graphs.

The project will focus on implementing a prototype of the system as an abstract
library, rather than lower-level components needed for real-world deployment,
such as the syntax of communications protocols or data formats. Producing the
latter is unnecessary, since many networks have such infrastructure already
that the library could hook onto (assuming that our assumptions and design
principles are compatible).


\section*{Evaluation}

I will measure how performance of the algorithm scales as it runs on different
network topologies and usage loads, and how it degrades as various parts of the
network fail or come under attack.

The prototype library should be light enough for up to perhaps 10,000 nodes to
be simulated on a single machine, which should be suitable for our purposes.

Realistic data can be obtained from social-networking sites. For this project
we need user network {\em and} tag network data. A brief look around the world
wide web shows that last.fm, flickr and delicious have such features, although
the data needs some further processing - eg. inferring weight and direction for
tag relationships, and working out which tags are more "general".

I will take increasing subgraphs of the final processed data set, and use it to
generate networks with statistical properties defined by various schemes. Some
interesting properties to experiment with might include:

\begin{description}
\item [intra-index properties:] eg. how semantically unified a given index is
    (ie. to what extent are its tags mutually related), or how specialised a
    given index is (ie. whether its tags are more "general" or "specific").
\item [inter-index properties:] eg. the cumulative distribution of indexes with
    varying levels of a given property, the types of neighbours an index with a
    given property is likely to link to.
\end{description}

A collection of search requests will be simulated on each network generated, to
test the performance of each scheme as it is applied to increasing network
sizes. Failures and attacks will also be simulated, eg. by having some indexes
deviate from the structure defined by the given scheme, giving false data, etc.


\section*{Success Criteria}

The project will have the following goals:

\begin{itemize}
\item To implement a versatile and extensible prototype of the system as a
    library.
\item To obtain a representative set of realistic data, and to process it (as
    described previously) into a form suitable to use for simulations.
\item To create a simulation suite that is flexible enough to run all the
    appropriate tests.
\item To obtain a comprehensive set of test results on the performance of the
    routing algorithm under various conditions.
\end{itemize}

A successful outcome is one that accomplishes all these goals. The technically
hardest part of the project will be to design a robust system, and to estimate
optimal performance conditions for it. The most time-consuming part will be to
design a comprehensive set of precise test conditions against which to evaluate
the system.

\section*{Resources Required}

This project is unlikely to require any additional resources other than my
personal computer. I will also use github as a backup service.


\section*{Timetable and Milestones}

As the project progresses, I will write much of the material required for the
dissertation as extended notes. This should reduce the time needed at the end
of the project to finalise it and prepare it for submission.

\begin{description}
\item [2009-11-w0 - 2009-12-w0:] Design the specific details of the system,
    including the routing algorithm. Read up on research into similar systems
    to get a general idea of the issues that need to be addressed.
\item [2009-12-w0 - 2009-12-w2:] Implement the library. This will include the
    data structures and algorithms designed in the previous step, as well as a
    high-level node class that could be deployed in a real system.
\item [2010-01-w1 - 2010-01-w3:] Design the test suite, including selecting an
    underlying storage network to use, and algorithms for populating a test
    network with data.
\item [2010-01-w3 - 2010-02-w0:] Implement the test suite. This will be a
    detachable adapter module between my library and the selected storage
    network.
\item [2010-02-w0 - 2010-02-w1:] Implement data extraction code. This will
    involve writing a crawler and a parser for one of the websites noted
    previously.
\item [2010-02-w1 - 2010-02-w3:] Design the set of properties for which to
    generate test data. During this time, the data extraction code will be
    running.
\item [2010-02-w3 - 2010-03-w0:] Implement data processing code. This will take
    the data extracted previously and convert it into a form appropriate for my
    system.
\item [2010-03-w0 - 2010-03-w2:] Implement test data generation code. During
    this time, the data processing code will be running.
\item [2010-03-w2 - 2010-04-w0:] Generate test data and run tests. This should
    be a fairly straightforward and automatic process; during this time I will
    gather and structure my notes into a format appropriate for my
    dissertation.
\item [2010-04-w0 - 2010-04-w2:] Evaluate results and add this to the
    dissertation.
\item [2010-04-w2 - 2010-05-w1:] Finalise and proof-read dissertation and
    prepare it for submission.
\end{description}

\end{document}

% Indexes could either be held locally by the maintainer, or published into the
% network. The exact details will be explored further in the project; the latter
% has the advantage that it can be implemented on top of *any* distributed storage
% medium that allows exact-match queries, such as a DHT.

% ...where to store tag-link data; within an index, or separate object...

% - need to convert this into a graph suitable for our system, ie. need direction and probably weight on each link
%   - in some cases, links are weighted using a bold font, but this is probably based
%     on the numerical output of some algorithm simple enough to reverse-engineer
%   - for unweighted links, we could make one up, such as {content with both tags} / {content with just one tag}

% This would also allow sharing of the more "general" indexes, which would perform
% better than everyone maintaining their own indexes, whilst retaining individual
% choice and flexibility (anyone can fork at any time).
